\documentclass[notheorems,mathserif,table,compress]{beamer}  %dvipdfm选项是关键,否则编译统统通不过
%%------------------------常用宏包------------------------
%%注意, beamer 会默认使用下列宏包: amsthm, graphicx, hyperref, color, xcolor, 等等
\usepackage{fontspec,xunicode,xltxtra}  % for XeTeX
\usepackage{comment}
\usepackage{fancybox}
\usepackage{tcolorbox}
%%------------------------ThemeColorFont------------------------
%% Presentation Themes
% \usetheme[<options>]{<name list>}
\usetheme{Madrid}
%% Inner Themes
% \useinnertheme[<options>]{<name>}
%% Outer Themes
% \useoutertheme[<options>]{<name>}
\useoutertheme{miniframes} 
%% Color Themes 
% \usecolortheme[<options>]{<name list>}
%% Font Themes
% \usefonttheme[<options>]{<name>}
\setbeamertemplate{background canvas}[vertical shading][bottom=white,top=structure.fg!7] %%背景色, 上25%的蓝, 过渡到下白.
\setbeamertemplate{theorems}[numbered]
\setbeamertemplate{navigation symbols}{}   %% 去掉页面下方默认的导航条.
\usepackage{zhfontcfg}

%\setsansfont[Mapping=tex-text]{文泉驿正黑}  %% 需要fontspec宏包
     %如果装了Adobe Acrobat,可在font.conf中配置Adobe字体的路径以使用其中文字体
     %也可直接使用系统中的中文字体如SimSun,SimHei,微软雅黑 等
     %原来beamer用的字体是sans family;注意Mapping的大小写,不能写错
     %设置字体时也可以直接用字体名,以下三种方式等同:
     %\setromanfont[BoldFont={黑体}]{宋体}
     %\setromanfont[BoldFont={SimHei}]{SimSun}
     %\setromanfont[BoldFont={"[simhei.ttf]"}]{"[simsun.ttc]"}
%%------------------------MISC------------------------
\graphicspath{{figures/}}         %% 图片路径. 本文的图片都放在这个文件夹里了.
%%------------------------正文------------------------
\begin{document}
\XeTeXlinebreaklocale "zh"         % 表示用中文的断行
\XeTeXlinebreakskip = 0pt plus 1pt % 多一点调整的空间
%%----------------------------------------------------------
%% This is only inserted into the PDF information catalog. Can be left
%% out.
%%%
%% Delete this, if you do not want the table of contents to pop up at
%% the beginning of each subsection:
\begin{comment}
\AtBeginSection[]{                              % 在每个Section前都会加入的Frame
  \frame<handout:0>{
    \frametitle{Content}\small
    \tableofcontents[current,currentsubsection]
  }
}
\AtBeginSubsection[]                            % 在每个子段落之前
{
  \frame<handout:0>                             % handout:0 表示只在手稿中出现
  {
    \frametitle{Contents}\small
    \tableofcontents[current,currentsubsection] % 显示在目录中加亮的当前章节
  }
}
\end{comment}
%%----------------------------------------------------------
\title[Hausdorff距离]{Hausdorff距离}
\author[wrc]{汇报人~~~~\textcolor{olive}{王如晨}}
%\hspace{2.28em}导师~~\textcolor{olive}{}~教授}
\institute[ouc]{\small\textcolor{violet}{中国海洋大学~~~~信息科学与工程学院}}
\date{2014年11月}
%\titlegraphic{\vspace{-6em}\includegraphics[height=7cm]{ouc}\vspace{-6em}}  %海大校徽
\frame{ \titlepage }
%%----------------------------------------------------------
%\section*{目录}
\frame{\frametitle{Contents}\tableofcontents}
%%----------------------------------------------------------

%\section{Beamer类和XeTeX概览} %如果你想书签不出现问题,请不要用\XeTeX
                                 %这类复杂的指令,直接写XeTeX吧
\section{经典Hausdorff距离}

%\subsection{是什么?}

\begin{frame}
\frametitle{Hausdorff距离\footnote{陈岚岚,毕笃彦,马时平.Hausdorff距离在图像匹配中的应用.现代电子技术.2002.}}
  %\begin{tcolorbox}[colback=red!5,colframe=blue!75!black]
   {\color{blue}\textbf{Hausdorff距离}}是描述两组点集之间相似程度的一种度量,是一种定义于两个点集上的最大最小距离。

~~~~~~若给定两个点集,$A=\{a_{1},a_{2},\cdots\}$,$B=\{b_{1},b_{2},\cdots\}$ ,则点集A、B之间的Hausdorff距离定义为:
\begin{displaymath}
H(A,B)=\max[h(A,B),h(B,A)]
\end{displaymath}
~~~~~~其中,$h(A,B)$称为集合A到B有向Hausdorff距离,即点集A中所有点到点集B的最小距离的最大值。
\begin{displaymath}
h(A,B)=\max_{\begin{subarray}{l}
              a\in A
               \end{subarray}}
       d_{B}(a)
~~~~
d_{B}(a)=\min_{\begin{subarray}{l}
              b\in B
               \end{subarray}}
        \parallel a-b \parallel
\end{displaymath}
  %\end{tcolorbox}
\end{frame}


%\subsection{部分Hausdorff距离(PHD)}

\begin{frame}
%\frametitle{基本流程}

   \begin{figure}[!ht]
    \begin{minipage}{0.4\textwidth}
    \centering
    \includegraphics[width=1.5in]{dian111.png}
    \caption{$a_{1}$到$B$的距离}
    \end{minipage}
    \begin{minipage}{0.4\textwidth}
    \centering
    \includegraphics[width=1.5in]{dian112.png}
    \caption{$a_{2}$到$B$的距离}
    \end{minipage}
   \end{figure}
\end{frame}

\begin{frame}
%\frametitle{基本流程}
   \begin{figure}[!ht]
    \centering
    \includegraphics[width=1.5in]{dian41.png}
    \caption{点集$A$到$B$的距离}
   \end{figure}
\end{frame}


\begin{frame}
%\frametitle{基本流程}

   \begin{figure}[!ht]
    \begin{minipage}{0.3\textwidth}
    \centering
    \includegraphics[width=1.5in]{dian223.png}
    \caption{$b_{1}$到$A$的距离}
    \end{minipage}
    \begin{minipage}{0.3\textwidth}
    \centering
    \includegraphics[width=1.5in]{dian222.png}
    \caption{$b_{2}$到$A$的距离}
    \end{minipage}
    \begin{minipage}{0.3\textwidth}
    \centering
    \includegraphics[width=1.5in]{dian221.png}
    \caption{$b_{3}$到$A$的距离}
    \end{minipage}
   \end{figure}
\end{frame}

\begin{frame}
%\frametitle{基本流程}

   \begin{figure}[!ht]
    \centering
    \includegraphics[width=1.5in]{dian42.png}
    \caption{点集$B$到$A$的距离}
   \end{figure}
\end{frame}

\begin{frame}
%\frametitle{基本流程}

   \begin{figure}[!ht]
    \centering
    \includegraphics[width=1.5in]{dian3.png}
    \caption{点集$A$、$B$之间的距离}
   \end{figure}
\begin{displaymath}
H(A,B)=\max[h(A,B),h(B,A)]
\end{displaymath}
\end{frame}

\begin{frame}
\frametitle{经典Hausdorff距离的缺点}
%  \begin{tcolorbox}[colback=red!5,colframe=blue!75!black]
~~~~~~受到噪声的影响较大。例如,即使A、B的形状相似,只要A中有一个点偏离B较远,那么计算出来的Hausdorff距离就会很大。
%  \end{tcolorbox}
\end{frame}

\section{部分Hausdorff距离(PHD)}

\begin{frame}
\frametitle{部分Hausdorff距离(PHD)\footnote{苏磊,张登福,王世强,刘涛.基于Hausdorff距离的图像匹配技术应用综述.第八届全国信号与信息处理联合学术议论文集.2009.}}
 % \begin{tcolorbox}[colback=red!5,colframe=blue!75!black]
{\color{blue}\textbf{部分Hausdorff距离}}的定义:
\begin{displaymath}
H_{K}(A,B)=\max[h_{K}(A,B),h_{K}(B,A)]
\end{displaymath}
\begin{displaymath}
h_{K}=K^{th}_{\begin{subarray}{l}
              a\in A
               \end{subarray}}
d_{B}(a)
\end{displaymath}
~~~~~~部分Hausdorff距离中,要求出点集A中所有点到点集B的距离,然后将这些距离从小到大排序,其中序号为K的距离就是$h_{K}(A,B)$。
\begin{displaymath}
K=[f\times N_{A}]
\end{displaymath}
~~~~~~部分hausdorff距离可以有效处理当目标发生遮蔽和有外部点存在的情形。
  %\end{tcolorbox}
\end{frame}

\section{改进Hausdorff距离(MHD)}   
 
\begin{frame}
\frametitle{改进Hausdorff距离(MHD)}
{\color{blue}\textbf{改进Hausdorff距离}}的定义:
\begin{displaymath}
H_{K}(A,B)=\max[h_{MHD}(A,B),h_{MHD}(B,A)]
\end{displaymath}
\begin{displaymath}
h_{MHD}(A,B)=\frac{1}{N_{A}}\sum_{a\in A} d_{B}(a)
\end{displaymath}
~~~~~~改进Hausdorff距离中,$h_{MHD}(A,B)$要求出点集A中所有点到点集B的距离,然后求取距离的平均值。
\end{frame}      

\section{LTS Hausdorff距离} 
\begin{frame}
\frametitle{LTS Hausdorff距离\footnote{沈大伟,段会川.基于LTS Hausdorff距离与遗传算法的图像配准方法.电子应用技术.2007.}}
{\color{blue}\textbf{LTS Hausdorff距离}}为了获得更加准确的目标匹配结果,结合上面两种方法得到的。
\begin{displaymath}
H_{LTS}(A,B)=\max[h_{LTS}(A,B),h_{LTS}(B,A)]
\end{displaymath}
\begin{displaymath}
h_{LTS}(A,B)=\frac{1}{H}\sum_{i=1}^{H} d_{B}(a)_{i}
\end{displaymath}
~~~~~~其中,$H=h\times N_{A},0<h<1$,$N_{A}$是点集$A$的个数,$d_{B}(a)_{i}$表示对序列$d_{B}(a)$排序后第i个距离。

~~~~~~LTS Hausdorff距离是一种排序再求部分均值的方法来确定点集A,B之间的距离。这种方法将噪声干扰和遮挡产生的大距离问题去除。
\end{frame}

\section{应用} 
\begin{frame}
\frametitle{应用}
 图像匹配\footnote{孙瑾,顾宏斌,秦小麟,周娜.一种鲁棒型 Hausdorff 距离图像匹配方法.中国图像图形学报.2008.}

   \begin{figure}[!ht]
    \begin{minipage}{0.2\textwidth}
    \centering
    \includegraphics[width=0.5in]{mo.png}
    \end{minipage}
    \begin{minipage}{0.2\textwidth}
    \centering
    \includegraphics[width=1in]{mu.png}
    \end{minipage}
   \end{figure}
   \begin{figure}[!ht]
    \centering
    \includegraphics[width=1in]{pi.png}
   \end{figure}
\end{frame}



\end{document}

