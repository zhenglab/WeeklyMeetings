\documentclass[notheorems,mathserif,table,compress]{beamer}  %dvipdfm选项是关键,否则编译统统通不过
%%------------------------常用宏包------------------------
%%注意, beamer 会默认使用下列宏包: amsthm, graphicx, hyperref, color, xcolor, 等等
\usepackage{fontspec,xunicode,xltxtra}  % for XeTeX
\usepackage{comment}
\usepackage{fancybox}
\usepackage{tcolorbox}
%%------------------------ThemeColorFont------------------------
%% Presentation Themes
% \usetheme[<options>]{<name list>}
\usetheme{Madrid}
%% Inner Themes
% \useinnertheme[<options>]{<name>}
%% Outer Themes
% \useoutertheme[<options>]{<name>}
\useoutertheme{miniframes} 
%% Color Themes 
% \usecolortheme[<options>]{<name list>}
%% Font Themes
% \usefonttheme[<options>]{<name>}
\setbeamertemplate{background canvas}[vertical shading][bottom=white,top=structure.fg!7] %%背景色, 上25%的蓝, 过渡到下白.
\setbeamertemplate{theorems}[numbered]
\setbeamertemplate{navigation symbols}{}   %% 去掉页面下方默认的导航条.
\usepackage{zhfontcfg}

%\setsansfont[Mapping=tex-text]{文泉驿正黑}  %% 需要fontspec宏包
     %如果装了Adobe Acrobat,可在font.conf中配置Adobe字体的路径以使用其中文字体
     %也可直接使用系统中的中文字体如SimSun,SimHei,微软雅黑 等
     %原来beamer用的字体是sans family;注意Mapping的大小写,不能写错
     %设置字体时也可以直接用字体名,以下三种方式等同:
     %\setromanfont[BoldFont={黑体}]{宋体}
     %\setromanfont[BoldFont={SimHei}]{SimSun}
     %\setromanfont[BoldFont={"[simhei.ttf]"}]{"[simsun.ttc]"}
%%------------------------MISC------------------------
\graphicspath{{figures/}}         %% 图片路径. 本文的图片都放在这个文件夹里了.
%%------------------------正文------------------------
\begin{document}
\XeTeXlinebreaklocale "zh"         % 表示用中文的断行
\XeTeXlinebreakskip = 0pt plus 1pt % 多一点调整的空间
%%----------------------------------------------------------
%% This is only inserted into the PDF information catalog. Can be left
%% out.
%%%
%% Delete this, if you do not want the table of contents to pop up at
%% the beginning of each subsection:
\begin{comment}
\AtBeginSection[]{                              % 在每个Section前都会加入的Frame
  \frame<handout:0>{
    \frametitle{Content}\small
    \tableofcontents[current,currentsubsection]
  }
}
\AtBeginSubsection[]                            % 在每个子段落之前
{
  \frame<handout:0>                             % handout:0 表示只在手稿中出现
  {
    \frametitle{Contents}\small
    \tableofcontents[current,currentsubsection] % 显示在目录中加亮的当前章节
  }
}
\end{comment}
%%----------------------------------------------------------
\title[形状匹配]{形状匹配}
\author[wrc]{汇报人~~~~\textcolor{olive}{王如晨}}
%\hspace{2.28em}导师~~\textcolor{olive}{}~教授}
\institute[ouc]{\small\textcolor{violet}{中国海洋大学~~~~信息科学与工程学院}}
\date{2014年10月}
%\titlegraphic{\vspace{-6em}\includegraphics[height=7cm]{ouc}\vspace{-6em}}  %海大校徽
\frame{ \titlepage }
%%----------------------------------------------------------
%\section*{目录}
\frame{\frametitle{Contents}\tableofcontents}
%%----------------------------------------------------------

%\section{Beamer类和XeTeX概览} %如果你想书签不出现问题,请不要用\XeTeX
                                 %这类复杂的指令,直接写XeTeX吧
\section{关于形状匹配}

\subsection{是什么?}

\begin{frame}
\frametitle{形状匹配是什么?}
  \begin{tcolorbox}[colback=red!5,colframe=blue!75!black]
   ~~~~在计算机视觉和模式识别中,形状通常由目标范围的二值图像所表示。它往往在目标识别中起重要作用,有时只靠形状我们就可以识别出物体。
   \begin{figure}[!ht]
    \centering
    \includegraphics[width=2in]{erzhitu.png}
   \end{figure} 
   {\color{blue}\textbf{形状匹配}}:
   就是按照一定的准则来衡量形状间的相似程度。它可以应用到目标识别、基于内容的图像检索(CBIR)、文字识别等领域。
  \end{tcolorbox}
\end{frame}


\subsection{基本流程}

\begin{frame}
\frametitle{基本流程}
   \begin{figure}[!ht]
    \centering
    \includegraphics[width=4in]{liuchengtu.png}
   \end{figure} 
   \begin{itemize}
    \item {\color{blue}\textbf{形状表示}}:用一些方法将物体的形状用数值的描述子表示。
    \item {\color{blue}\textbf{相似度计算}}:就是在形状描述的基础上计算两个目标间的相似度,根据相似度的大小进行匹配。形状相似度的值越大,表示两个形状越相似。有些形状匹配方法是求取形状间的{\color{blue}{非相似度}}通常它被称为{\color{blue}{形状间的距离}}。
   \end{itemize}
\end{frame}


\section{形状表示的方法}

\begin{frame}
\frametitle{形状表示的方法}
  \begin{tcolorbox}[colback=red!5,colframe=blue!75!black]
   ~~~~形状特征描述的方法很多,根据获得的形状特征的方法不同,可以分为{\color{blue}\textbf{基于轮廓的形状表示方法}}和{\color{blue}\textbf{基于区域的形状表示方法}}。前一种方法的形态特征来自于物体的轮廓曲线信息,后一种方法来自于形状区域信息。在这个基础上,还可以进一步的分为全局描述法和结构化描述法。
  \end{tcolorbox}
\end{frame}

\begin{frame}
\frametitle{基于轮廓的形状表示方法}
  \begin{itemize}
   \item 结构化方法
	\begin{itemize}
	 \item 链码
	 \item 多边形近似
	 \item 样条
	 \item 不变量
	\end{itemize}
   \item 全局方法
	\begin{itemize}
	 \item 周长
	 \item 曲率尺度空间
	 \item fourier描述子
	 \item 小波描述子
	 \item 形状上下文
	 \item 形态描述子
	 \item Hough变换
	 \item 自回归模型
	 \item 弹性匹配
	\end{itemize}
  \end{itemize}
\end{frame}



\begin{frame}
\frametitle{基于区域的形状表示方法}
  \begin{itemize}
   \item 结构化方法
	\begin{itemize}
	 \item 凸壳
	 \item 骨架(中轴)
	 \item 形状核
	\end{itemize}
   \item 全局方法
	\begin{itemize}
	 \item 面积
	 \item 偏心率
	 \item 正交矩
	 \item 几何代数不变矩
	 \item 栅格方法
	 \item 广义Fourier变换
	 \item 形状矩阵
	 \item 欧拉数
	 \item Zernike矩
	 \item Legendre矩
	\end{itemize}
  \end{itemize}
\end{frame}

\begin{frame}
\begin{itemize}
\item 形状上下文\\ 
~~~~~~一种基于统计学的新方法。将形状轮廓看作一个轮廓点集的基础上,考察轮廓序列上某个指定参考点与该轮廓序列上其他所有参考点的空间位置分布关系。统计轮廓点集中每一个特征点与其他特征点的位置关系,得到对应的形状直方图。\\ 
~~~~~~优点:充分利用上下文信息,在非刚性物体匹配中有较好的鲁棒性。\\
~~~~~~缺点:对与背景噪声点过多的情况,匹配效果不好。 
\end{itemize}
\end{frame}

\begin{frame}
  \begin{itemize}
   \item 凸壳\\ 
  \end{itemize}
~~~~~~是指包含一个平面点集的最小凸区域。通过凸壳算法可以计算出形状的凹凸结构,使用凹凸结构提取形状特征,递归生成一个凹状树来描述物体的形状。这样将形状匹配问题转化为图匹配问题。\\ 
   \begin{figure}[!ht]
    \centering
    \includegraphics[width=3in]{tuke.png}
   \end{figure}
   \begin{figure}
    \begin{minipage}{0.3\textwidth}
    \centering
    \includegraphics[width=1.2in]{tukeapple.png}
    \end{minipage}
    \begin{minipage}{0.3\textwidth}
    \centering
    \includegraphics[width=1in]{tuke3.png}
    \end{minipage}
   \end{figure}
\end{frame}

\begin{frame}
  \begin{itemize}
   \item 骨架\\ 
  \end{itemize}
   ~~~~~~也称为中轴,可以定义为一组互相连接的沿着形状各个分支的中线的集合。它的基本想法就是除去冗余信息,只保留有助于识别的关于物体结构的拓扑信息,用图来描述形状,将形状描述转换为图描述。\\
   \begin{figure}[!ht]
    \centering
    \includegraphics[width=1in]{gujia1.png}
   \end{figure}
\end{frame}

\begin{frame}
  \begin{tcolorbox}[colback=red!5,colframe=blue!75!black]
~~~~~~但是现在存在的骨架算法很多对边界噪声和变化极为敏感,因此需要对骨架进行剪枝。
   \begin{figure}[!ht]
    \centering
    \includegraphics[width=4in]{gujia.png}
   \end{figure}
剪枝算法分为:(1)依据骨架点的重要程度来判断其是否要保留;(2)在骨架化之前对物体边界进行光滑。 
  \end{tcolorbox}
\end{frame}

\begin{frame}
  \begin{tcolorbox}[colback=red!5,colframe=blue!75!black]
~~~~~~基于轮廓的形状表示缺失了形状内部的信息,对复杂形状表示能力有限。而基于区域的形状表示考察轮廓内所有的像素,可以更好的考察形状表现出来的特征。但它的局限性在于它们丢失了很多重要的形状细节。因此,有在轮廓表示的基础上,将基于区域的形状表示方法和基于轮廓的表示方法结合,可以提升对形状的表示能力。将形状的整体特性与局部特性结合来进行形状描述已经成为主流思路。
  \end{tcolorbox}
\end{frame}



\section{相似度计算方法}   
 
\begin{frame}
\frametitle{相似度计算方法}
  \begin{itemize}
   \item 欧式距离
   \item Hausdorff距离
   \item 二次式距离
   \item 直方图相交法
   \item 马氏距离
  \end{itemize}
\end{frame}      

\begin{frame}
  \begin{itemize}
   \item 欧式距离\\
计算两点之间的距离,是最简单的一种计算相似度的方法。它的物理意义清晰、计算简单、适用范围广。但是对于非均匀颜色空间里的两个像素点之间的距离不太好用。\\
   \item Hausdorff距离\\
用来计算两组点集之间的距离的一种相似度计算方法,计算的是两组点集上的最大-最小距离。常常用在二值图像或轮廓曲线的匹配中。\\
   \item 二次式距离\\
计算彩色图像间的相似度时,考虑不同颜色之间的相似度,在基于彩色直方图的图匹配比欧式距离效果更好。\\
  \end{itemize}
\end{frame}

\begin{frame}
\frametitle{形状匹配的难点}
  难点问题主要包括:
  \begin{enumerate}
   \item 物体的自遮挡和互遮挡
   \item 非刚性物体的运动
   \item 同类物体间的变化
   \item 图像噪声和分割错误
  \end{enumerate}
\end{frame}

\begin{frame}
  \begin{tcolorbox}[colback=red!5,colframe=blue!75!black]
~~~~~~目前基于轮廓和基于骨架的形状表示成为研究的主流,而轮廓点集空间分布关系和多尺度表示是基于轮廓的形状表示的主要研究方向。基于点集的形状匹配和图匹配也成为研究的重点。
  \end{tcolorbox}
\end{frame}

\end{document}

