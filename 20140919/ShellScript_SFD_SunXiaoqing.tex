\documentclass[notheorems,mathserif,table,compress,10pt]{beamer}  %dvipdfm选项是关键,否则编译统统通不过
%%------------------------常用宏包------------------------
%%注意, beamer 会默认使用下列宏包: amsthm, graphicx, hyperref, color, xcolor, 等等
\usepackage{fontspec,xunicode,xltxtra}  % for XeTeX
%%------------------------ThemeColorFont------------------------
%% Presentation Themes
% \usetheme[<options>]{<name list>}
\usetheme{Madrid}
%% Inner Themes
% \useinnertheme[<options>]{<name>}
%% Outer Themes
% \useoutertheme[<options>]{<name>}
\useoutertheme{miniframes} 
%% Color Themes 
% \usecolortheme[<options>]{<name list>}
%% Font Themes
\usefonttheme[small]{serif}
\setbeamertemplate{background canvas}[vertical shading][bottom=white,top=structure.fg!7] %%背景色, 上25%的蓝, 过渡到下白.
\setbeamertemplate{theorems}[numbered]
\setbeamertemplate{navigation symbols}{}   %% 去掉页面下方默认的导航条.
\usepackage{zhfontcfg}
\usepackage{iplouclistings}
\usepackage{iplouccfg}
\usepackage{tcolorbox}
%\setsansfont[Mapping=tex-text]{文泉驿正黑}  %% 需要fontspec宏包
     %如果装了Adobe Acrobat,可在font.conf中配置Adobe字体的路径以使用其中文字体
     %也可直接使用系统中的中文字体如SimSun,SimHei,微软雅黑 等
     %原来beamer用的字体是sans family;注意Mapping的大小写,不能写错
     %设置字体时也可以直接用字体名,以下三种方式等同:
     %\setromanfont[BoldFont={黑体}]{宋体}
     %\setromanfont[BoldFont={SimHei}]{SimSun}
     %\setromanfont[BoldFont={"[simhei.ttf]"}]{"[simsun.ttc]"}
%%------------------------MISC------------------------
\graphicspath{{figures/}}         %% 图片路径. 本文的图片都放在这个文件夹里了.
%%------------------------正文------------------------
\begin{document}
\XeTeXlinebreaklocale "zh"         % 表示用中文的断行
\XeTeXlinebreakskip = 0pt plus 1pt % 多一点调整的空间
%%----------------------------------------------------------
%% This is only inserted into the PDF information catalog. Can be left
%% out.
%%%
%% Delete this, if you do not want the table of contents to pop up at
%% the beginning of each subsection:
\AtBeginSection[]{                              % 在每个Section前都会加入的Frame
  \frame<handout:0>{
    \frametitle{内容}\small
    \tableofcontents[current,currentsubsection]
  }
}

\AtBeginSubsection[]                            % 在每个子段落之前
{
  \frame<handout:0>                             % handout:0 表示只在手稿中出现
  {
    \frametitle{Content}\small
    \tableofcontents[current,currentsubsection] % 显示在目录中加亮的当前章节
  }
}

%%----------------------------------------------------------
%\title{中国海常见有害赤潮藻显微图像识别}
\title{Shell 脚本那些事}
%\subtitle{~~~~-想和你分享}
\author[CVBIOUC]{~~\textcolor{olive}{孙晓庆 \quad王如晨}\\
    \quad \textcolor{olive}{CVBIOUC}\\
    \textcolor{blue}{\url{http://vision.ouc.edu.cn/~zhenghaiyong}
    \\
    \href{mailto:sunxiaoqingouc@gmail.com}{sunxiaoqingouc@gmail.com}}
    }
%\author{\textcolor{olive}{Sun xiaoqing}}
    %\quad Major~~\textcolor{olive}{电子与通信工程}
%\institute[中国海洋大学]{\small\textcolor{violet}{中国海洋大学~~信息科学与工程学院}}
\date{2014~年~9~月~19~日}
%\titlegraphic{\vspace{-6em}\includegraphics[height=7cm]{ouc}\vspace{-6em}}
\frame{ \titlepage }
%%----------------------------------------------------------
\section*{内容}
\frame{\frametitle{内容}\tableofcontents}
%%----------------------------------------------------------

\section{初遇shell 脚本 }

\begin{frame}
\frametitle{因实验需要而使用}
\begin{tcolorbox}[colback=blue!5,colframe=blue!75!black]

 我对5万张放在嵌套文件夹下的图像进行实验,实验后每张图像生成了三个文件,其中一个是.txt文件。为了对实验数据分析,我需要把这5万个txt 文件的数据合并到一个总的.txt文件中,并记录每个.txt数据的路径。
\end{tcolorbox}

\end{frame}  

\begin{frame}
\frametitle{面对这个问题}
解决方案?
\begin{itemize}
\item  手动
\item  用Matlab 或C语言编写程序实现
\item  简单好用的方法----- shell 脚本
\end{itemize}
\end{frame}  


 \begin{bash}
 #!/bin/bash  
 INPUTDIR=$1
 OUTPUTTXT=$2
 num=1
 for i in $(find ${INPUTDIR} -type f -name "*.txt"|grep -v "TimeNumber.txt"); do  
  line2=`sed -n '2p' $i`
  echo -e "$((num++))\t$i\t${line2}" >> $OUTPUTTXT
  done
 \end{bash}
 
 \begin{frame}
\frametitle{Shell 脚本也可以“高大上”}
\begin{tcolorbox}[colback=blue!5,colframe=blue!75!black]
  对脚本做了完善 
\end{tcolorbox}
\end{frame}  
 
 \begin{bash}
#!/bin/bash

if [ $# != 2 ]; then
	echo -ne "Arguments Error.\n"
	echo -ne "\tUsage:\n"
	echo -ne "\t$0 <INPUTDIR> <OUTPUTTXT>\n"
	exit 7
fi

INPUTDIR=$1
OUTPUTTXT=$2
num=1

if [ -f ${OUTPUTTXT} ]; then
	rm -f ${OUTPUTTXT}
fi

for i in $(find ${INPUTDIR} -type f -name "*.txt"|grep -v "TimeNumber.txt"); do
	echo -ne "$((num++)): Processing \033[41;33m $i \033[0m" 
	line2=`sed -n '2p' $i`
	echo -e "$((num-1))\t$i\t${line2}" >> $OUTPUTTXT
	echo -ne "\t DONE!\n"
done
echo -ne "\n"
echo -ne "\t\tEND PROCESSING $((num-1)) TEXTS!\n"
\end{bash}

 \begin{frame}
\frametitle{其它的批处理问题}
\begin{itemize}
\item  两个文件夹下分别有 1000 幅图片,其图片名不一样,但内容可能一样, 编写脚本找到内容一样的图像。
\item 某文件夹下有 100 个 pdf 格式的文件,文件按数字 1-100 命名,如果要 在中间加入一个文件 27.pdf,那么原文件夹中文件从 27 开始就要依次 加 1。
\item  将某文件夹下的所有文件(扩展名都相同)重新按数字命名。
\item  要在某目录下的所有子文件夹中新建 test 的文件夹。
\end{itemize}
\end{frame}  



 \section{再用shell 脚本}
\begin{frame}
\frametitle{实验的自动操作 ----- 追求真正的简单 }
\begin{itemize}
\item  {\color{blue}\textbf{ 写实验的自动操作shell 脚本前}}
\begin{itemize}
\item打开matlab运行程序。
\item 在终端调用脚本MergeTXT.sh,把结果合并成总的 .txt文件。
\item  在终端用$gnuplot$对总的.txt文件里的数据画图分析。 
\item  在实验过程中,我需要多次调参数,需要多次进行实验。需要多次重复上述过程,花费很多时间。
\end{itemize}
\item  {\color{blue}\textbf{ 写实验的自动操作shell 脚本后}}
\begin{itemize}
\item  只需要设定好参数值的范围,在终端直接运行 shell 脚本。
\end{itemize}
\end{itemize}
\end{frame}

 \begin{bash}
 #!/bin/bash

MatlabEXE='/Applications/MATLAB_R2013a.app/bin/matlab'
InputDir='/Users/sunxiaoqing/Desktop/FSD/RunTest'
BinDir='/Users/sunxiaoqing/bin'
MatlabFileDir="${InputDir}/MatlabProgram/test.m"
MatlabResultDir="${InputDir}/MatlabResult"
MergeTXTResultDir="${InputDir}/MergeTXTResult/"
GnuplotResultDir="${InputDir}/GnuplotResult/"
StartNum=20
StartGnuplotNum=20

chmod +x $MatlabFlieDir
${MatlabEXE} -nodesktop -nosplash  -r "run ${MatlabFlieDir};quit"

for i in $(find ${MatlabResultDir} -name 'result*');  do
         j="${MergeTXTResultDir}/result$((StartNum++)).txt"
         source "${BinDir}/MergeTXT.sh"  $i $j
done
for k in $(find ${MergeTXTResultDir} -type f -name "*.txt"); do
         GnuplotFilename="${GnuplotResultDir}""result""$((StartGnuplotNum++))"".jpg"
         source "${BinDir}/PlotTXTdata.sh" $GnuplotFilename  $k
done
 \end{bash}


\section{Shell 脚本}

\begin{frame}
\frametitle{Shell 脚本是什么}
 \begin{itemize}
 \item Shell 脚本语言包括了所有linux 命令,以及同时还有类似于高级程序设计语言的语法结构,如分支判断语句、变量和函数、循环结构、数组、算术和逻辑运算等。
  \item  Shell 脚本:为了实现特定功能,用脚本语言编写的一个程序。

\end{itemize}
\end{frame}



\begin{frame}
\frametitle{Shell 脚本的用途}
 \begin{itemize}
 \item 文件的批处理
\item  电脑中的重要文件的自动备份 
\item  将服务器上的资料同步到本地
\item   连续命令单一化
\item   对主机的管理
\item  $\cdots\cdots$ 
\end{itemize}
\end{frame}

\begin{frame}
\frametitle{Shell 脚本的好处}
 \begin{itemize}
  \item 简单、易学、易用。
 \item 处理复杂的批处理问题。
 \item 调用系统已有的资源,去实现一些特定的功能。
 \item 可移植性比较好。
 \end{itemize}
\end{frame}

\begin{frame}
\frametitle{如果你也对Shell脚本感兴趣}
 \begin{itemize}
  \item Shell 十三问
 \item https://github.com/zhenghaiyong/Scripts

\end{itemize}
\end{frame}

\begin{frame}
\begin{tcolorbox}[colback=blue!5,colframe=blue!75!black]
感谢我们的郑老师,带我们走进Linux!
\end{tcolorbox}
\begin{figure}\centering 
  \begin{minipage}[t]{0.7\linewidth} 
    
    \includegraphics[width=\linewidth]{team1.jpg}
%    \label{fig:} 
  \end{minipage}
\end{figure} 
\end{frame}



\begin{frame}

\begin{figure}\centering 
  \begin{minipage}[t]{0.7\linewidth} 
    
    \includegraphics[width=\linewidth]{thank.png}
%    \label{fig:} 
  \end{minipage}
\end{figure} 
\end{frame}


\end{document}
