\documentclass[notheorems,mathserif,table,compress]{beamer}  %dvipdfm选项是关键,否则编译统统通不过
%%------------------------常用宏包------------------------
%%注意, beamer 会默认使用下列宏包: amsthm, graphicx, hyperref, color, xcolor, 等等
\usepackage{fontspec,xunicode,xltxtra}  % for XeTeX
\usepackage{comment}
\usepackage{fancybox}
\usepackage{tcolorbox}
\usepackage{wrapfig}
%%------------------------ThemeColorFont------------------------
%% Presentation Themes
% \usetheme[<options>]{<name list>}
\usetheme{Madrid}
%% Inner Themes
% \useinnertheme[<options>]{<name>}
%% Outer Themes
% \useoutertheme[<options>]{<name>}
\useoutertheme{miniframes} 
%% Color Themes 
% \usecolortheme[<options>]{<name list>}
%% Font Themes
% \usefonttheme[<options>]{<name>}
\setbeamertemplate{background canvas}[vertical shading][bottom=white,top=structure.fg!7] %%背景色, 上25%的蓝, 过渡到下白.
\setbeamertemplate{theorems}[numbered]
\setbeamertemplate{navigation symbols}{}   %% 去掉页面下方默认的导航条.
\usepackage{zhfontcfg}

%\setsansfont[Mapping=tex-text]{文泉驿正黑}  %% 需要fontspec宏包
     %如果装了Adobe Acrobat,可在font.conf中配置Adobe字体的路径以使用其中文字体
     %也可直接使用系统中的中文字体如SimSun,SimHei,微软雅黑 等
     %原来beamer用的字体是sans family;注意Mapping的大小写,不能写错
     %设置字体时也可以直接用字体名,以下三种方式等同:
     %\setromanfont[BoldFont={黑体}]{宋体}
     %\setromanfont[BoldFont={SimHei}]{SimSun}
     %\setromanfont[BoldFont={"[simhei.ttf]"}]{"[simsun.ttc]"}
%%------------------------MISC------------------------
\graphicspath{{figures/}}         %% 图片路径. 本文的图片都放在这个文件夹里了.
%%------------------------正文------------------------
\begin{document}
\XeTeXlinebreaklocale "zh"         % 表示用中文的断行
\XeTeXlinebreakskip = 0pt plus 1pt % 多一点调整的空间
%%----------------------------------------------------------
%% This is only inserted into the PDF information catalog. Can be left
%% out.
%%%
%% Delete this, if you do not want the table of contents to pop up at
%% the beginning of each subsection:
\begin{comment}
\AtBeginSection[]{                              % 在每个Section前都会加入的Frame
  \frame<handout:0>{
    \frametitle{Content}\small
    \tableofcontents[current,currentsubsection]
  }
}
\AtBeginSubsection[]                            % 在每个子段落之前
{
  \frame<handout:0>                             % handout:0 表示只在手稿中出现
  {
    \frametitle{Contents}\small
    \tableofcontents[current,currentsubsection] % 显示在目录中加亮的当前章节
  }
}
\end{comment}
%%----------------------------------------------------------
\title[]{Mathematical Problems from VALSE Webinars}
\author[]{}
%\hspace{2.28em}导师~~\textcolor{olive}{}~教授}
\institute[ouc]{\small\textcolor{violet}{}}
\date{}
%\titlegraphic{\vspace{-6em}\includegraphics[height=7cm]{ouc}\vspace{-6em}}  %海大校徽
\frame{ \titlepage }
%%----------------------------------------------------------
%\section*{目录}
\frame{\frametitle{Contents}\tableofcontents}
%%----------------------------------------------------------

%\section{Beamer类和XeTeX概览} %如果你想书签不出现问题,请不要用\XeTeX

\begin{frame}
paperreading中提到的数学知识:
\begin{itemize}
\item 优化问题
\item 基本数学概念
\item 统计问题
\end{itemize}
\end{frame}

\section{优化问题}
\begin{frame}
\frametitle{优化问题}
优化问题在这些论文中有很广泛的应用,其涉及两个方面:
\begin{itemize}
\item 优化问题的给出
\item 优化问题的求解
\end{itemize}
\end{frame}

\begin{frame}
\frametitle{Separable Kernel for Image Deblurring\footnote{L. Fang, H. Liu, F. Wu, X. Sun, H. Li. Separable Kernel for Image Deblurring. CVPR. 2014.}}
这篇文章主要考虑单张图片由于相机运动造成的模糊去除问题
\begin{figure}[t]
    \begin{tabular}{ccccc}
    \begin{minipage}[t]{1in}
    \includegraphics[width=1in]{BlurImage.png}
    \end{minipage}
%%    
    \begin{minipage}[t]{0.1in}
    \includegraphics[width=0.1in]{equal.png}
    \end{minipage}

    \begin{minipage}[t]{1in}
    \includegraphics[width=1in]{LatentSharpImage.png}
    \end{minipage}
    \begin{minipage}[t]{0.1in}
    \includegraphics[width=0.1in]{con.png}
    \end{minipage}
    \begin{minipage}[t]{1in}
    \includegraphics[width=0.5in]{BlurKernel.png}
    \end{minipage}
\end{tabular}
\end{figure}

\begin{equation}
B=K\otimes I+N
\end{equation}
\end{frame}


\begin{frame}
\frametitle{Separable Kernel for Image Deblurring}
\begin{wrapfigure}{l}{0.5\textwidth}
  \vspace{-10pt}
  \begin{center}
    \includegraphics[width=0.5\textwidth]{SBK}
  \end{center}
  \vspace{-20pt}
  \vspace{-10pt}
\end{wrapfigure}
Separable Blur Kernel:
\begin{itemize}
\item \textbf{Trajectory}\\(projection of camera shake in 2D image plane)
\item \textbf{Intensity}\\(staying time of shaking camera in every position)
\item \textbf{Point Spread Function}\\(decided by camera focus scene depth and camera motion at the perpendicular direction of image plane)
\end{itemize}
\end{frame}

\begin{frame}
\frametitle{Separable Kernel for Image Deblurring}
优化函数:
\begin{equation}
\begin{array}{c}
K_p^*= arg \min_{K_p,I_p}||\nabla B_p-\nabla I_p\otimes K_p||_2^2+\lambda_1\frac{\nabla ||I_p||_1}{\nabla ||I_p||_2}+\lambda_2||W\circ K_p||_1
\\s.t.\quad W=1-G(T^*_p)
\end{array}
\end{equation}
\end{frame}

\begin{frame}
\frametitle{Linearized Alternating Direction Method: Two Blocks and Multiple Blocks\footnote{Zhouchen Lin, Risheng Liu, and Zhixun Su, Linearized Alternating Direction Method with Adaptive Penalty for Low Rank Representation, NIPS 2011, arXiv: 1109.0367.}}
很多计算机视觉的优化问题可归结为如下模型:
\begin{equation}
\begin{array}{cc}
\min_{x_1,x_2} & f(x_1)+f(x_2)\\
s.t. & \mathcal{A}_1(x_1)+\mathcal{A}_2(x_2)=b
\end{array}
\end{equation}
\end{frame}

\begin{frame}
\frametitle{Linearized Alternating Direction Method: Two Blocks and Multiple Blocks}

增广拉格朗日函数:
\begin{equation}
\begin{array}{ccc}
\mathcal{L}(x_1,x_2,\lambda) & = & f_1(x_1)+f_2(x_1)+<\lambda,\mathcal{A}_1(x_1)+\mathcal{A}_2(x_2)-b>\\
& + & \frac{\beta}{2}||\mathcal{A}_1(x_1)+\mathcal{A}_2(x_2)-b||_F^2
\end{array}
\end{equation}
\end{frame}

\begin{frame}
\frametitle{Linearized Alternating Direction Method: Two Blocks and Multiple Blocks}

交替迭代法求解:
\begin{equation}
\begin{array}{ccc}
x_1^{k+1} & = & arg \min_{x_1} \mathcal{L}(x_1,x_2^k,\lambda^k)\\
x_2^{k+1} & = & arg \min_{x_2} \mathcal{L}(x_1^k,x_2,\lambda^k)\\
\lambda^{k+1} & = & \lambda^k+\beta_k[\mathcal{A}_1(x_1^{k+1})+\mathcal{A}_2(x_2^{k+1})-b]
\end{array}
\end{equation}
\end{frame}

\begin{frame}
\frametitle{Linearized Alternating Direction Method: Two Blocks and Multiple Blocks}

等价与下面的形式(对两种形式求导会发现得到相同的结果):
\begin{equation}
\begin{array}{ccc}
x_1^{k+1} & = & arg \min_{x_1} f(x_1)+\frac{\beta_k}{2}||\mathcal{A}_1(x_1)+\mathcal{A}_2(x_2^k)-b+\frac{\lambda_k}{\beta_k}||^2\\
x_2^{k+1} & = & arg \min_{x_2} f(x_2)+\frac{\beta_k}{2}||\mathcal{A}_1(x_1^{k+1})+\mathcal{A}_2(x_2)-b+\frac{\lambda_k}{\beta_k}||^2
\end{array}
\end{equation}
\end{frame}

\begin{frame}
\frametitle{Linearized Alternating Direction Method: Two Blocks and Multiple Blocks}
二次项做泰勒展开:
\begin{eqnarray}
x_1^{k+1} & = & arg \min_{x_1} f(x_1)+<\mathcal{A}_1^*(\lambda_k+\beta_k(\mathcal{A}_1(x_1^k)+\mathcal{A}_2(x_2^k)-b)),x_1-x_1^k>\\
\nonumber
&&+\frac{\beta_k\eta_1}{2}||x_1-x_1^k||^2\\
& = & arg \min_{x_1} f(x_1)
\\
\nonumber
&& +\frac{\beta_k\eta_1}{2}||x_1-x_1^k+\frac{\mathcal{A}_1^*(\lambda_k+\beta_k(\mathcal{A}_1(x_1^k)+\mathcal{A}_2(x_2^k)-b))}{\beta_k\eta_1}||^2
\end{eqnarray}
\end{frame}

\section{基本数学概念}
\begin{frame}
\frametitle{基本数学概念}
\begin{itemize}
\item 距离计算
\item 投影
\item ...
\end{itemize}
\end{frame}

\begin{frame}
\frametitle{Metric Learning Driven Multi-Task Structured Output Optimization for Robust Keypoint Tracking\footnote{Metric Learning-Driven Multi-Task Structured Output Optimization for Robust Keypoint Tracking,” Proceedings of Twenty-Ninth AAAI Conference on Artificial Intelligence (AAAI), 2015}}
\begin{figure}
\includegraphics[width=3in]{TemInput}
\end{figure}
\begin{center}
distance($d_i$,$d_j$)
\end{center}
\begin{equation}
\left[ 
\begin{array}{c}
d_1\\
d_2\\
\vdots\\
d_{N_1}
\end{array}
\right]
\longrightarrow
\left[ 
\begin{array}{c}
d^*_1\\
d^*_2\\
\vdots\\
d^*_{N_2}
\end{array}
\right]
\end{equation}
\end{frame}

\section{统计问题}
\begin{frame}
\frametitle{统计问题}
\begin{itemize}
\item 统计建模
\item EM算法
\item 联合分布
\item 转移概率
\item ...
\end{itemize}
\end{frame}

\begin{frame}
\frametitle{Matrix Factorization with Unknown Noise\footnote{Deyu Meng, Fernando De la Torre. Robust Matrix Factorization with Unknown Noise. International Conference of Computer Vision (ICCV), 2013.}}

\begin{figure}
\includegraphics[width=2in]{tong3}
\end{figure}
\begin{equation}
X=UV^T+\epsilon
\end{equation}
\end{frame}

\begin{frame}
\frametitle{Matrix Factorization with Unknown Noise}
\begin{figure}
\includegraphics[width=3in]{tong1}
\end{figure}
\end{frame}

\begin{frame}
\frametitle{Matrix Factorization with Unknown Noise}
模型:
\begin{equation}
x_{ij}=u_i^Tv_j+\epsilon_{ij}\quad p(\epsilon)\sim \sum_{k=1}^{K}\pi_k\aleph(0,\sigma_k^2)
\end{equation}
\end{frame}

\begin{frame}
\frametitle{Matrix Factorization with Unknown Noise}
\begin{figure}
\includegraphics[width=3in]{tong2}
\end{figure}
\end{frame}



\begin{frame}
\begin{center}
谢谢!
\end{center}
\end{frame}
\end{document}

