\documentclass[notheorems,mathserif,table,compress,10pt]{beamer}  %dvipdfm选项是关键,否则编译统统通不过
%%------------------------常用宏包------------------------
%%注意, beamer 会默认使用下列宏包: amsthm, graphicx, hyperref, color, xcolor, 等等
\usepackage{fontspec,xunicode,xltxtra}  % for XeTeX
%%------------------------ThemeColorFont------------------------
%% Presentation Themes
% \usetheme[<options>]{<name list>}
\usetheme{Madrid}
%% Inner Themes
% \useinnertheme[<options>]{<name>}
%% Outer Themes
% \useoutertheme[<options>]{<name>}
\useoutertheme{miniframes} 
%% Color Themes 
% \usecolortheme[<options>]{<name list>}
%% Font Themes
\usefonttheme[small]{serif}
\setbeamertemplate{background canvas}[vertical shading][bottom=white,top=structure.fg!7] %%背景色, 上25%的蓝, 过渡到下白.
\setbeamertemplate{theorems}[numbered]
\setbeamertemplate{navigation symbols}{}   %% 去掉页面下方默认的导航条.
\usepackage{zhfontcfg}
\usepackage{iplouccfg}
%\setsansfont[Mapping=tex-text]{文泉驿正黑}  %% 需要fontspec宏包
     %如果装了Adobe Acrobat,可在font.conf中配置Adobe字体的路径以使用其中文字体
     %也可直接使用系统中的中文字体如SimSun,SimHei,微软雅黑 等
     %原来beamer用的字体是sans family;注意Mapping的大小写,不能写错
     %设置字体时也可以直接用字体名,以下三种方式等同:
     %\setromanfont[BoldFont={黑体}]{宋体}
     %\setromanfont[BoldFont={SimHei}]{SimSun}
     %\setromanfont[BoldFont={"[simhei.ttf]"}]{"[simsun.ttc]"}
%%------------------------MISC------------------------
\graphicspath{{figures/}}         %% 图片路径. 本文的图片都放在这个文件夹里了.
%%------------------------正文------------------------
\begin{document}
\XeTeXlinebreaklocale "zh"         % 表示用中文的断行
\XeTeXlinebreakskip = 0pt plus 1pt % 多一点调整的空间
%%----------------------------------------------------------
%% This is only inserted into the PDF information catalog. Can be left
%% out.
%%%
%% Delete this, if you do not want the table of contents to pop up at
%% the beginning of each subsection:
\AtBeginSection[]{                              % 在每个Section前都会加入的Frame
  \frame<handout:0>{
    \frametitle{Content}\small
    \tableofcontents[current,currentsubsection]
  }
}

\AtBeginSubsection[]                            % 在每个子段落之前
{
  \frame<handout:0>                             % handout:0 表示只在手稿中出现
  {
    \frametitle{Content}\small
    \tableofcontents[current,currentsubsection] % 显示在目录中加亮的当前章节
  }
}

%%----------------------------------------------------------
%\title{中国海常见有害赤潮藻显微图像识别}
\title{关于login shell和 non-login shell 的几个问题 }
%\subtitle{~~~~~-A Week }
\author[OUC]{Name:~~~~~\textcolor{olive}{Sun Xiaoqing}\\
    \quad Major:~~\textcolor{olive}{Electronics and Communication Engineering}}
%\author{\textcolor{olive}{Sun xiaoqing}}
    %\quad Major~~\textcolor{olive}{电子与通信工程}
%\institute[中国海洋大学]{\small\textcolor{violet}{中国海洋大学~~信息科学与工程学院}}
\date{September 12, 2014 ~}
%\titlegraphic{\vspace{-6em}\includegraphics[height=7cm]{ouc}\vspace{-6em}}
\frame{ \titlepage }
%%----------------------------------------------------------
\section*{Content}
\frame{\frametitle{Content}\tableofcontents}
%%----------------------------------------------------------



\section{什么是login shell和 non-login shell?}

\begin{frame}
\frametitle{什么是shell?}
\begin{itemize}
\item shell 叫做壳。它接收用户命令,然后调用相应的应用程序。
\end{itemize}
\end{frame}  

\begin{frame}
\frametitle{login shell和 non-login shell}
\begin{itemize}
\item  {\color{blue}\textbf{ login shell}}:我们开机后,首次登录的终端就是一个login  shell。
\item  {\color{blue}\textbf{ non-login shell}}:在第二次及以后登录终端,这个bash 环境都是一个non-login shell。
\end{itemize}
\end{frame}  

\section{login shell和 non-login shell的区别?}
\begin{frame}
\frametitle{login shell和 non-login shell的区别:系统读取的配置文件不同}
\begin{itemize}
\item  {\color{blue}\textbf{ login shell}}系统会先读取/etc/profile, 然后再依次读取$\sim$$/.bash\_profile$ 或$\sim$$/.bash\_login$ 或$\sim$$/.proflie$ 这些配置文件。
\begin{itemize}
\item  /etc/profile是系统文件,最好不要轻易更改这个文件。
\item $\sim$$/.bash\_profile$ 或$\sim$$/.bash\_login$ 或$\sim$$/.proflie$这些配置文件属于用户自己的,如果你想要添加路径的话,在这些文件里添加就可以。
\end{itemize}
\item  {\color{blue}\textbf{ non-login shell}}系统会自动的读取$\sim$$/.bashrc$中的配置文件。
\end{itemize}
 \end{frame}



\begin{frame}
\frametitle{对于ubuntu14.04}
如果 login shell登录时,是不是无法读取到$\sim$$/.bashrc$中的文件呢?
\begin{figure}\centering 
  \begin{minipage}[t]{0.9\linewidth} 
    
    \includegraphics[width=1\linewidth]{profile.png}
%    \label{fig:} 
\caption{$\sim$$/.proflie$}
  \end{minipage} 
   \end{figure} 
 为了保证login shell 时也可以读取到 non-login shell 中的配置文件。
 \end{frame}
\begin{frame}

\frametitle{安装软件的路径添加在哪里?}
比如以我们安装\LaTeX{} 为例:
\begin{itemize}
\item  /etc/profile是系统文件,最好不要把路径添加这个文件。
\item 把路径添加到  $\sim$$/.proflie$下,当我们login shell 时,这些路径就开始生效。
\item 把路径添加在 $\sim$$/.bashrc$ 下,在你login shell和 non-login shell 时,路径都会生效。(个人觉得添加在这里最好)
\end{itemize}
 \end{frame}

\section{其他的一些 login shell 和non-login shell}
\begin{frame}
\frametitle{login shell和 non-login shell}
\begin{itemize}
\item 登录系统时获得的顶层shell,无论是通过本地终端登录,还是通过网络ssh登录。这种情况下获得的login shell是一个交互式shell。
\item  在终端下使用$--$$login$选项调用bash,可以获得一个交互式login shell。
\item  在脚本中使用$--$$login$选项调用bash,(比如在shell脚本第一行做如下指定:\#!/bin/bash - -login),此时得到一个非交互式的login shell。
\item 使用"$su$ $-$"切换到指定用户时,获得此用户的login shell。如果不使用"$-$",则获得$non-login shell$。 
\end{itemize}
\end{frame}  

\begin{frame}
\frametitle{疑问?}
\begin{itemize}
\item 为什么会有 login shell 和non-login shell? 这样设计的初衷是为什么?
\end{itemize}
\end{frame}  

\begin{frame}
\begin{figure}\centering 
  \begin{minipage}[t]{0.7\linewidth} 
    \includegraphics[width=\linewidth]{thank.png}
%    \label{fig:} 
  \end{minipage}
\end{figure} 
\end{frame}


\end{document}
