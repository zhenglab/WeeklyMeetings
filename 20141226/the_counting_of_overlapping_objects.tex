\documentclass[notheorems,mathserif,table,compress]{beamer}  %dvipdfm选项是关键,否则编译统统通不过
%%------------------------常用宏包------------------------
%%注意, beamer 会默认使用下列宏包: amsthm, graphicx, hyperref, color, xcolor, 等等
\usepackage{fontspec,xunicode,xltxtra}  % for XeTeX
\usepackage{comment}
\usepackage{fancybox}
%%------------------------ThemeColorFont------------------------
%% Presentation Themes
% \usetheme[<options>]{<name list>}
\usetheme{Madrid}
%% Inner Themes
% \useinnertheme[<options>]{<name>}
%% Outer Themes
% \useoutertheme[<options>]{<name>}
\useoutertheme{miniframes} 
%% Color Themes 
% \usecolortheme[<options>]{<name list>}
%% Font Themes
% \usefonttheme[<options>]{<name>}
\setbeamertemplate{background canvas}[vertical shading][bottom=white,top=structure.fg!7] %%背景色, 上25%的蓝, 过渡到下白.
\setbeamertemplate{theorems}[numbered]
\setbeamertemplate{navigation symbols}{}   %% 去掉页面下方默认的导航条.
\usepackage{zhfontcfg}
%\setsansfont[Mapping=tex-text]{文泉驿正黑}  %% 需要fontspec宏包
     %如果装了Adobe Acrobat,可在font.conf中配置Adobe字体的路径以使用其中文字体
     %也可直接使用系统中的中文字体如SimSun,SimHei,微软雅黑 等
     %原来beamer用的字体是sans family;注意Mapping的大小写,不能写错
     %设置字体时也可以直接用字体名,以下三种方式等同:
     %\setromanfont[BoldFont={黑体}]{宋体}
     %\setromanfont[BoldFont={SimHei}]{SimSun}
     %\setromanfont[BoldFont={"[simhei.ttf]"}]{"[simsun.ttc]"}
%%------------------------MISC------------------------
\graphicspath{{figures/}}         %% 图片路径. 本文的图片都放在这个文件夹里了.
%%------------------------正文------------------------
\begin{document}
\XeTeXlinebreaklocale "zh"         % 表示用中文的断行
\XeTeXlinebreakskip = 0pt plus 1pt % 多一点调整的空间
%%----------------------------------------------------------
%% This is only inserted into the PDF information catalog. Can be left
%% out.
%%%
%% Delete this, if you do not want the table of contents to pop up at
%% the beginning of each subsection:
\begin{comment}
\AtBeginSection[]{                              % 在每个Section前都会加入的Frame
  \frame<handout:0>{
    \frametitle{Content}\small
    \tableofcontents[current,currentsubsection]
  }
}
\AtBeginSubsection[]                            % 在每个子段落之前
{
  \frame<handout:0>                             % handout:0 表示只在手稿中出现
  {
    \frametitle{下一节内容}\small
    \tableofcontents[current,currentsubsection] % 显示在目录中加亮的当前章节
  }
}
\end{comment}
%%----------------------------------------------------------
\title[]{The Counting of Overlapping Objects}
\author[戴嘉伦]{\textcolor{olive}{戴嘉伦}}
  %\hspace{2.28em}导师~~\textcolor{olive}{姬光荣}~教授}
\institute[中国海洋大学]{\small\textcolor{violet}{中国海洋大学}}
\date{\today}
%\titlegraphic{\vspace{-6em}\includegraphics[height=7cm]{ouc}\vspace{-6em}}
\frame{ \titlepage }
%%----------------------------------------------------------
%\section*{目录}
\frame{\frametitle{目录}\tableofcontents}
%%----------------------------------------------------------

%\section{Beamer类和XeTeX概览} %如果你想书签不出现问题,请不要用\XeTeX
                                 %这类复杂的指令,直接写XeTeX吧
\section{应用:计数}
\begin{frame}
  \frametitle{应用}
   在实际应用中,对于计数问题,需要注重重叠问题。\\
   \hspace{0.2in}方法:
   \begin{itemize}
	\item 不分割重叠物体进行计数
	\item 分割重叠物体进行计数
   \end{itemize}
\end{frame}

\section{不分割重叠}
\begin{frame}
    \frametitle{不分割重叠}
    重叠藻类计数\footnote{武宗茜,王鹏,丁天怀,活动轮廓模型在重叠藻细胞计数中的应用,计算机工程,2012.Vol.38,No.3}:
    \begin{itemize}
       \item 采用活动轮廓模型算法提取重叠藻细胞边缘,进行边缘分析和计数
       \item 活动轮廓模型具有检测精度高、易建模、适合提取任意形状的变形轮廓且简单的优点
    \end{itemize}
    步骤:
    \begin{itemize}
       \item 对二值图进行连通域进行连通域标记,每完成一次标记,记录连通域外接矩形坐标和连通域面积
       \item 运用活动轮廓模型分割,对目标图像进行分割,获得目标边缘坐标,保存在矩阵中,根据矩阵判断该边缘是否为封闭边缘
       \item 根据封闭边缘个数的不同,来分辨藻细胞个数完成计数
    \end{itemize}
\end{frame}

\begin{frame}
    \begin{figure}
   \begin{minipage}[t]{0.4\textwidth} 
     \centering 
     \includegraphics[width=1.2in]{juxing.png} \\
     外接矩形
   \end{minipage}
   \begin{minipage}[t]{0.4\textwidth} 
     \centering 
     \includegraphics[width=1.0in]{zao1.png} \\
     封闭边缘为1
   \end{minipage}
   \begin{minipage}[b]{0.4\textwidth} 
     \centering 
     \includegraphics[width=1.0in]{zao2.png} \\
     封闭边缘为2
   \end{minipage}
   \begin{minipage}[b]{0.4\textwidth} 
     \centering 
     \includegraphics[width=1.0in]{zao3.png} \\
     封闭边缘为3
   \end{minipage}
   \end{figure}
\end{frame}



\begin{frame}
    \frametitle{不分割重叠}
   重叠细胞识别\footnote{刘斌,基于奇异值分解的显微图像重叠细胞识别,2005,吉林大学硕士学位论文}:\\
   \begin{itemize}
       \item 二维自适应阈值分割方法
        \begin{itemize}
	   \item 图像像素灰度和像素点领域平均灰度的二维直方图阈值
	   \item 利用了图像像素与其领域的空间相关信息,更强的抗噪声能力
	\end{itemize}
       \item 边界剥离方法进一步分割
	\begin{itemize}
	   \item 对重叠细胞区域进行层层剥离,判断是否发生了细胞分裂
	   \item 若发生分裂,在刚剥离的那层边界上,搜索局部分离点,从而在原图上搜索分离点;否则继续剥离
	   \item 避免一般分离算法要求重叠细胞连接处的凹陷性比较明显
	   \item 要求在细胞连接处存在局部最小灰度值
	\end{itemize}
    \end{itemize}
\end{frame}

\begin{frame}
   \begin{itemize}
	\item 细胞识别
	\begin{itemize}
	    \item 利用特征向量空间,采用奇异值特征方法,对细胞进行奇异值分解,用结果矩阵表示一个细胞
   	    \item 通过真实样本的图像矩阵的奇异值特征向量,计算图像归属度,得到阈值
	    \item 求待测图像矩阵的奇异值特征向量,计算其图像归属度,通过与阈值的比较,判断待测图像为哪一类图像
	    \item 比传统的纹理分析方法、灰度方差阵等方法有更好的效果和更准确的识别率
	\end{itemize}
   \end{itemize}
   \begin{figure}
     \begin{minipage}[b]{0.4\textwidth} 
     \centering 
     \includegraphics[width=1.0in]{shibie1.png} \\
     识别结果(a)
   \end{minipage}
   \begin{minipage}[b]{0.4\textwidth} 
     \centering 
     \includegraphics[width=1.1in]{shibie2.png} \\
     识别结果(b)
   \end{minipage}
   \end{figure}
\end{frame}


\begin{frame}
    \frametitle{不分割重叠}
    显微图像分割\footnote{王祥生,王伟,基于清晰度的细胞显微图像分割和计数方法,中国医学影像学杂志,2014,Vol.2,No.10:797-800}:
    \begin{itemize}
        \item 对图像灰度化,并进行离散余弦变换(DCT)变换,低频部分表示图像轮廓,高频部分表示细节特性
     	\item 将低频信号值一半作为阈值,截断高频信号,进行反DCT变换到灰度空间,与原灰度图像求差并灰度化
    \end{itemize}
     \begin{figure}
     \begin{minipage}[b]{0.4\textwidth} 
     \centering 
     \includegraphics[width=1.4in]{dct1.png} \\
     (a)
   \end{minipage}
   \begin{minipage}[b]{0.4\textwidth} 
     \centering 
     \includegraphics[width=1.4in]{dct2.png} \\
     (b)
   \end{minipage}
   \end{figure}
\end{frame}

\begin{frame}
    \begin{itemize}
    	\item 自适应阈值法将差值图像二值化,消除离散噪声点,进行区域生长
    	\item 连通区域标记,在区域外接矩形坐标核连通区域面积,通过聚类的方法,得到两类
    	\item 将两类中面积较小的一类的面积中值作为单个细胞面积,估计其他连通区域包含的细胞个数,得到总体的细胞个数
    \end{itemize}
     \begin{figure}
     \begin{minipage}[b]{0.4\textwidth} 
     \centering 
     \includegraphics[width=1.4in]{dct3.png} \\
     (c)
   \end{minipage}
   \begin{minipage}[b]{0.4\textwidth} 
     \centering 
     \includegraphics[width=1.4in]{dct4.png} \\
     (d)
   \end{minipage}
   \end{figure}
\end{frame}

\section{分割重叠}
\begin{frame}
    \frametitle{分割重叠}
    \begin{itemize}
        \item 分水岭算法:对不同区域设计标记,循环标记结果形成分水岭
        \item 形态学方法:利用腐蚀操作找到分割点,并实现重叠颗粒的分割
        \item 凹点匹配:利用凹点来描述边界的凹陷情况(常用)
    \end{itemize}
\end{frame}

\begin{frame}
    \frametitle{凹点匹配\footnote{韦冬冬,基于凹点匹配的重叠图像分割算法,计算机与应用化学,2010,Vol.27,No.1}}
    \begin{itemize}
	\item 启发式搜索算法:从重叠区域凹点序列中取一组起始点和结束点
	\item 凹点序列中的最大曲率点作为起始点,将距离该起始点最近的1个凹点作为终止点,实现重叠区域的分割
	\item 根据重叠情况使用不同凹点匹配方法,根据匹配凹点之间的距离是否小于给定的阈值来判断是否匹配
    \end{itemize}
\end{frame}

\begin{frame}
    P为当前边界点,M为前继点,N为后继点\\
    通过向量的夹角与阈值比较,判断凹点是否为分离点对
     \begin{figure}
     \begin{minipage}[b]{0.5\textwidth} 
     \centering 
     \includegraphics[width=2.0in]{aodian1.png} \\
     凹点判断(a)
   \end{minipage}
   \begin{minipage}[b]{0.5\textwidth} 
     \centering 
     \includegraphics[width=1.9in]{aodian2.png} \\
     凹点判断(b)
   \end{minipage}
   \end{figure}
\end{frame}

\section{补全}
\begin{frame}
    \frametitle{检测方法}
    \begin{itemize}
	\item 主动轮廓模型方法(Active Contour Method)
	\item Mumford-shah Function: \newline
	\begin{displaymath}
	E_i(u,C)=\frac{1}{2} \int_{\Omega} (f-u)^2 dx + 
        \frac{\lambda^2}{2} \int_{\Omega-C} \lvert \triangledown u\rvert ^2 dx +
	v \int_{0}^{1} C_s^2 ds 
	\end{displaymath}
    \end{itemize}
\end{frame}

\begin{frame}
   \begin{Huge}
   \textbf{Thank you very much!}
   \end{Huge}
\end{frame}


\end{document}
