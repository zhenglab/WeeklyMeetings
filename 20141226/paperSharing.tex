\documentclass[notheorems,serif,table,compress]{beamer}  %dvipdfm选项是关键,否则编译统统通不过
%%------------------------常用宏包------------------------
%%注意, beamer 会默认使用下列宏包: amsthm, graphicx, hyperref, color, xcolor, 等等
\usepackage{fontspec,xunicode,xltxtra}  % for XeTeX
\usepackage{verbatim}
\usepackage{mathabx}
\usepackage{latexsym}
\usepackage{amsfonts,amssymb}
\usepackage{iplouclistings}
\usepackage{fancybox}
\usepackage{colortbl}
\usepackage{tcolorbox}
%\usepackage[T1]{fontenc}
%\usepackage{bookman}
\usepackage{subfigure}
\usepackage{hyperref}
\usepackage{listings}
\usepackage{animate}
\usepackage[absolute,overlay]{textpos}
\usepackage{graphicx}
\usepackage{tikz}
\usepackage[americaninductors,europeanresistors]{circuitikz}

%%------------------------ThemeColorFont------------------------
%% Presentation Themes
% \usetheme[<options>]{<name list>}
%\usetheme{Madrid}
\usetheme{Berkeley}
%% Inner Themes双精度计算
% \useinnertheme[<options>]{<name>}
%% Outer Themes
% \useoutertheme[<options>]{<name>}
%\useoutertheme{miniframes} 
%% Color Themes 
%\usecolortheme[<options>]{<name list>}
%% Font Themes
\usefonttheme{serif}
\setbeamertemplate{background canvas}[vertical shading][bottom=white,top=structure.fg!7] %%背景色, 上25%的蓝, 过渡到下白.
\setbeamertemplate{theorems}[numbered]
\setbeamertemplate{navigation symbols}{}   %% 去掉页面下方默认的导航条.
\usepackage{zhfontcfg}
%\setsansfont[Mapping=tex-text]{文泉驿正黑}  %% 需要fontspec宏包
     %如果装了Adobe Acrobat,可在font.conf中配置Adobe字体的路径以使用其中文字体
     %也可直接使用系统中的中文字体如SimSun,SimHei,微软雅黑 等
     %原来beamer用的字体是sans family;注意Mapping的大小写,不能写错
     %设置字体时也可以直接用字体名,以下三种方式等同:
     %\setromanfont[BoldFont={黑体}]{宋体}
     %\setromanfont[BoldFont={SimHei}]{SimSun}
     %\setromanfont[BoldFont={"[simhei.ttf]"}]{"[simsun.ttc]"}
%%------------------------MISC------------------------
\graphicspath{{figures/}}         %% 图片路径. 本文的图片都放在这个文件夹里了.
%%------------------------listing------------------------
\lstset{language=[LaTeX]TeX,Python}
%%------------------------正文------------------------
\begin{document}
\XeTeXlinebreaklocale "zh"         % 表示用中文的断行
\XeTeXlinebreakskip = 0pt plus 1pt % 多一点调整的空间
%%----------------------------------------------------------
%% This is only inserted into the PDF information catalog. Can be left
%% out.
%%%
%% Delete this, if you do not want the table of contents to pop up at
%% the beginning of each subsection:
%\AtBeginSection[]{                              % 在每个Section前都会加入的Frame
%  \frame<handout:0>{
%    \frametitle{Contents}\small
%    \tableofcontents[current,currentsubsection]
%  }
%}
%
%\AtBeginSubsection[]                            % 在每个子段落之前
%{
%  \frame<handout:0>                             % handout:0 表示只在手稿中出现
%  {
%    \frametitle{Contents}\small
%    \tableofcontents[current,currentsubsection] % 显示在目录中加亮的当前章节
%  }
%}

%%----------------------------------------------------------
\logo{\includegraphics[scale=0.13]{logol.png}}
\title{Paper Sharing}
\subtitle{Bottom-Up Saliency Detection Model Based on Human Visual Sensitivity and Amplitude Spectrum}
\author[]{\textcolor{olive}{Zhao Hongmiao}}
\institute[CVBIOUC]
{
\small\textcolor{violet}{CVBIOUC\\
Ocean University of China\\
\url{http://vision.ouc.edu.cn/~zhenghaiyong}}
}
\date[December 26, 2014]{December 26, 2014}
\titlegraphic{
\includegraphics[height=1.0cm]{ouc-logo.png}}
\frame{ \titlepage }
%%----------------------------------------------------------
%\section*{Contents}
\frame{\frametitle{Contents}\tableofcontents}
%%----------------------------------------------------------
\def\hilite<#1>{\temporal<#1>{\color{blue!15}}{\color{black}}{\color{black}}}
\newcommand{\shadow}[2][purple]{\hskip5pt\shadowbox{\color{#1}\small \kai #2\vspace{3mm}}}
\newcommand{\colorrbox}[2][purple]{\doublebox{\color{#1}\small \kai#2}}

%============================================================================

\section{Introduction}

\begin{frame}
\frametitle{Introduction}
\begin{itemize}
\item Author
%\tcblower
\begin{itemize}
\begin{columns}
\begin{column}{\leftmargini}
\end{column}
\begin{column}{.5\linewidth}
\item {Fang Yuming}
\item 1984.10.
\item lecturer 
\item Jiangxi University of Finance and Economics
\end{column}
\begin{column}{0.5\linewidth}

\begin{figure}
 \includegraphics[width=1.8cm]{FangYuming}
\end{figure}

\end{column}
\end{columns}\vspace{1ex}
\end{itemize}

\item Citation
\begin{itemize}
\item IEEE TRANSACTIONS ON MULTIMEDIA
\item Impact Factor: 1.776
\item Citation Rate: 31
\end{itemize}

\item Highlights
\begin{itemize}
\item The human visual sensitivity and QFT amplitude spectrum
\item Application: image retargeting
\end{itemize}
\end{itemize}
\end{frame}

%==========================================================================
\section{Model}

\begin{frame}
\frametitle{Saliency Detection Model}
\begin{figure}
 \includegraphics[width=4.5cm]{framework}
 \caption{Framework of the proposed saliency detection model.}
\end{figure}

\end{frame}

%--------------------------------------------------------------------------
\begin{frame}
\frametitle{Saliency Value for Each Patch}
\begin{itemize} 
\item Two factors
\begin{itemize} 
\item The patch differences between this image patch and all other image patches in the input image$\Longrightarrow$ {\color{blue}$\mathcal{D}_{(i,j)}$}
\item  The weighting for these patch differences$\Longrightarrow$ {\color{blue}$\alpha_{ij}$}
\end{itemize}
\item Definition
\begin{itemize} 
\begin{tcolorbox}[colback=blue!5,colframe=blue!75!black] 
\item $S_{i}=\sum_{j\neq i}\alpha_{ij}\mathcal{D}_{(i,j)}$ 
\end{tcolorbox}
\end{itemize}
\end{itemize}

\end{frame}

%--------------------------------------------------------------------------
\begin{frame}
\frametitle{Differences Between Image Patches and Their Weighting to Saliency Value}
\begin{itemize} 
\item Differences Between Image Patches
\begin{itemize} 
\item Quaternion Fourier Transform
\item $\mathcal{D}_{(i,j)}=\sqrt{\sum_{m}(log(\mathcal{A}_{m}^{i}+1)-log(\mathcal{A}_{m}^{j}+1))^{2}}$
\end{itemize}
\item The weight for the patch difference between different patches\footnote{W.S.~Geisler and J.S.~Perry, “A real-time foveated multi-solution system for low-bandwidth video communication,” in Proc. SPIE, Jul. 1998, vol. 3299, pp. 294–305.} 
\begin{equation}
\mathcal{C}_{s}(f,e)=\frac{1}{\mathcal{C}_{t}(f,e)}
\end{equation}
\begin{equation}
\mathcal{C}_{t}(f,e)=\mathcal{C}_{0}exp(\alpha f\frac{e+e_{2}}{e_{2}})
\end{equation}
\begin{equation}
e=tan^{-1}(\frac{d}{v})
\end{equation}
\begin{equation}
\alpha_{ij}=\frac{1}{\mathcal{C}_{0}exp(\alpha f\frac{e+e_{2}}{e_{2}})}
\end{equation}
\end{itemize}

\end{frame}

%------------------------------------------------------------------------------
\begin{frame}
\frametitle{Patch Size and Scale for Final Saliency Value}
\begin{itemize} 
\item Patch Size
\begin{figure}
 \includegraphics[width=6cm]{PatchSize}
 \caption{Original images and its different saliency maps with different patch sizes.}
\end{figure}
\begin{itemize}
\item The smaller image patch size, the more accurate saliency map we can get.
\item Suitable patch size: consideration of {\color{blue}fovea characteristics}, {\color{blue}saliency detection performance}, and {\color{blue}the computational complexity}.
\end{itemize}

\end{itemize}

\end{frame}

%------------------------------------------------------------------------------
\begin{frame}
\frametitle{Patch Size and Scale for Final Saliency Value}
\begin{itemize} 
\item Scale
\begin{itemize}
\item the steerable pyramid algorithm
\item $S_{i}=\frac{1}{N} \sum_{k}S_{i}^{k}$
\end{itemize}
\end{itemize}

\end{frame}

%================================================================================
\section{Evaluations}

\begin{frame}
\frametitle{Evaluations}
\begin{itemize}
\item Database: MSRA\footnote{T. Liu, J. Sun, N. Zheng, X. Tang, and H. Y. Shum, “Learning to detect a salient object,” in Proc. IEEE Int. Conf. Computer Vision and Pattern Recognition, 2007.} 
\item Comparison methods: PO, SR, MD, FG (temporary abbreviation)
\end{itemize}
\begin{figure}
 \includegraphics[width=4.5cm]{comparison}
 \caption{Experiment results for the comparison between the proposed model and others.}
\end{figure}

\end{frame}

%=================================================================================
\section{Conclusions}

\begin{frame}
\frametitle{Conclusions}
\begin{itemize}
\item {\color{blue}Advantages}
\begin{itemize}
\item Image Patch $\Longrightarrow$ Superpixels
\item $\mathcal{D}_{(i,j)}$ is novel
\item Steerable pyramid algorithm
\end{itemize}
\item {\color{blue}Disadvantages}
\begin{itemize}
\begin{columns}
\begin{column}{\leftmargini}
\end{column}
\begin{column}{.5\linewidth}
\item Accuracy
\item Computational complexity
\end{column}
\begin{column}{0.5\linewidth}

\begin{figure}
 \includegraphics[width=2.5cm]{ManHua}
\end{figure}

\end{column}
\end{columns}\vspace{1ex}
\end{itemize}
\end{itemize}

\end{frame}

%===============================================================================
\section{Thinking}
\begin{frame}
\frametitle{Thinking}
\begin{itemize}
\item Conclusion in frequency domain
\begin{textblock*}{10cm}(2cm,5.5cm)
\rowcolors[]{1}{blue!20}{blue!10}
\begin{tabular}{cc|cc}
Model & Method & Model & Method \\
SR & Residual Spectrum & HFT & SSS  \\
PFT & Only Phase Spectrum & SDSP & log-Gabor  \\
SIG & DCT+Sign & FT & DoG
\end{tabular}
\end{textblock*}
\end{itemize}
\end{frame}

%=================================================================================
\begin{frame}
\begin{figure}\centering
\begin{minipage}[t]{1\linewidth}
\includegraphics[width=\linewidth]{thank.jpg}
\end{minipage}
\end{figure}

\end{frame}


\end{document}
