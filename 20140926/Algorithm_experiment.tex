\documentclass[notheorems,mathserif,table,compress,dvipsnames]{beamer}  %dvipdfm选项是关键,否则编译统统通不过
%%------------------------常用宏包------------------------
%%注意, beamer 会默认使用下列宏包: amsthm, graphicx, hyperref, color, xcolor, 等等
\usepackage{fontspec,xunicode,xltxtra}  % for XeTeX
\usepackage{verbatim}
\usepackage{mathabx}
\usepackage{amsfonts,amssymb}
\usepackage{iplouclistings}
\usepackage{color}
\usepackage{fancybox}
%%------------------------ThemeColorFont------------------------
%% Presentation Themes
% \usetheme[<options>]{<name list>}
\usetheme{Madrid}
%% Inner Themes双精度计算
% \useinnertheme[<options>]{<name>}
%% Outer Themes
% \useoutertheme[<options>]{<name>}
\useoutertheme{miniframes} 
%% Color Themes 
%\usecolortheme[<options>]{<name list>}
%% Font Themes
\usefonttheme{serif}
\setbeamertemplate{background canvas}[vertical shading][bottom=white,top=structure.fg!7] %%背景色, 上25%的蓝, 过渡到下白.
\setbeamertemplate{theorems}[numbered]
\setbeamertemplate{navigation symbols}{}   %% 去掉页面下方默认的导航条.
\usepackage{zhfontcfg}
%\setsansfont[Mapping=tex-text]{文泉驿正黑}  %% 需要fontspec宏包
     %如果装了Adobe Acrobat,可在font.conf中配置Adobe字体的路径以使用其中文字体
     %也可直接使用系统中的中文字体如SimSun,SimHei,微软雅黑 等
     %原来beamer用的字体是sans family;注意Mapping的大小写,不能写错
     %设置字体时也可以直接用字体名,以下三种方式等同:
     %\setromanfont[BoldFont={黑体}]{宋体}
     %\setromanfont[BoldFont={SimHei}]{SimSun}
     %\setromanfont[BoldFont={"[simhei.ttf]"}]{"[simsun.ttc]"}
%%------------------------MISC------------------------
\graphicspath{{figures/}}         %% 图片路径. 本文的图片都放在这个文件夹里了.
%%------------------------正文------------------------
\begin{document}
\XeTeXlinebreaklocale "zh"         % 表示用中文的断行
\XeTeXlinebreakskip = 0pt plus 1pt % 多一点调整的空间
%%----------------------------------------------------------
%% This is only inserted into the PDF information catalog. Can be left
%% out.
%%%
%% Delete this, if you do not want the table of contents to pop up at
%% the beginning of each subsection:
%\AtBeginSection[]{                              % 在每个Section前都会加入的Frame
%  \frame<handout:0>{
%    \frametitle{Contents}\small
%    \tableofcontents[current,currentsubsection]
%  }
%}

%\AtBeginSubsection[]                            % 在每个子段落之前
%{
%  \frame<handout:0>                             % handout:0 表示只在手稿中出现
%  {
%    \frametitle{Contents}\small
%    \tableofcontents[current,currentsubsection] % 显示在目录中加亮的当前章节
%  }
%}
%
%%----------------------------------------------------------
\title{Partial Differential Equation and Image Processing}
\subtitle{~~~~~-A Week Summary}
\author[Qiu]{\textcolor{olive}{QiuXinxin}}
\institute[OUC]{\small\textcolor{violet}{Ocean University of China}\\
\small\textcolor{violet}{College of Information Science and Engineering}}
\date{September 26, 2014}
%\titlegraphic{\vspace{-6em}\includegraphics[height=7cm]{ouc}\vspace{-6em}}
\frame{ \titlepage }
%%----------------------------------------------------------
%\section*{Contents}
%\frame{\frametitle{Solving Systems of Linear Equations}\tableofcontents}
%%----------------------------------------------------------

%
\begin{frame}
\frametitle{Peak signal-to-noise ratio(PSNR) }
\begin{itemize}
\item Wiener, L-R, Regularized, Deconvblind, NNCGM
\item The PSNR, defined by
\begin{displaymath}
PSNR=10\log_{10}\bigg(\frac{(2^n-1)^2}{MSE}\bigg)
\end{displaymath}
MSE-MeanSquareError
\end{itemize}
\end{frame}

%
\begin{frame}
\frametitle{Test nncgm algorithm}
\begin{table}
\begin{tabular}{|c|c|c|c|c|c|c|} 
\hline
sigma&image& \multicolumn{1}{|c|}{Wiener}&LR&Regularized&Decon&NNCGM\\
\cline{3-7}
&& psnr&psnr&psnr&psnr&psnr\\
%sigma&image&Wiener&LR&Regularized&Deconvblind&NNCGM\\
\hline
%& &psnr&&&&\\  
%\hline
15&satellite&3.6086&22.1536&21.8713&23.8652&26.1569\\
\hline
30&satellite&3.9532&22.8697&23.7966&26.2953&27.8230\\
\hline
45&satellite&3.9358&22.8958&23.7578&26.0631&27.8472\\
\hline
%45&7&12&&&&\\
%\hline
%NNCGM&&&&&&\\
%\hline
\end{tabular}
\caption{Result with PSF=5X5, noisevar=0.001}
\end{table}
\end{frame}

%
\begin{frame}
\frametitle{ }
\begin{figure}[!ht]
\includegraphics[width=5.2in]{satelite-5x5-15-0001.jpg}
\caption{result with PSF=5X5, sigma=15, noisevar=0.001}
\end{figure}
\end{frame}

%
\begin{frame}
\frametitle{}
\begin{table}
\begin{tabular}{|c|c|c|c|c|c|c|} 
\hline
sigma&image& \multicolumn{1}{|c|}{Wiener}&LR&Regularized&Decon&NNCGM\\
\cline{3-7}
&& psnr&psnr&psnr&psnr&psnr\\
\hline
15&license&4.9973&19.8336&29.2548&25.4769&21.5675\\
\hline
30&license&5.0245&19.7909&27.2769&25.4637&21.4644\\
\hline
45&license&5.0195&19.7975&27.2420&25.5416&21.4976\\
\hline
\end{tabular}
\caption{Result with PSF=5X5, noisevar=0.01}
\end{table}
\end{frame}

%
\begin{frame}
\frametitle{ }
\begin{figure}[!ht]
\includegraphics[width=5.2in]{license-5x5-40-001.jpg}
\caption{result with PSF=5X5, sigma=40, noisevar=0.01}
\end{figure}
\end{frame}

%
\begin{frame}
\frametitle{}
\begin{table}
\begin{tabular}{|c|c|c|c|c|c|c|} 
\hline
psf&image& \multicolumn{1}{|c|}{Wiener}&LR&Regularized&Decon&NNCGM\\
\cline{3-7}
&& psnr&psnr&psnr&psnr&psnr\\
\hline
5X5&satellite&3.6140&15.5458&18.6767&20.4977&16.2139\\
\hline
9X9&satellite&3.4887&15.5262&17.3443&18.9824&16.9114\\
\hline
\end{tabular}
\caption{Result with sigma=15, noisevar=0.001}
\end{table}
\end{frame}


%
\begin{frame}
\frametitle{ }
\begin{figure}[!ht]
\includegraphics[width=5.2in]{satellite-9x9-15-0001.jpg}
\caption{result with PSF=9X9, sigma=15, noisevar=0.001}
\end{figure}
\end{frame}

%
\begin{frame}
\frametitle{ }
\begin{figure}[!ht]
\includegraphics[width=5.2in]{board-5x5-45-0001.jpg}
\caption{result with PSF=5X5, sigma=45, noisevar=0.001}
\end{figure}
\end{frame}


\end{document}
